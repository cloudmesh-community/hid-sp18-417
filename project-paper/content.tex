% status: 2
% chapter: Security

\title{Financial Analysis Service}

\author{Rashmi Ray}
\affiliation{%
  \institution{Indiana University}
  \streetaddress{107 S. Indiana Avenue}
  \city{Bloomington}
  \state{Indiana}
  \postcode{43017-6221}
}
\email{rashray@iu.edu}
% The default list of authors is too long for headers}

\renewcommand{\shortauthors}{Uma Kugan}

\begin{abstract}
The project involves creating a Kubernete cluster in GoogleCloud and later 
automating the installation and setup process by including the shell script.
The same process will be executed in a Kubernetes Chameleon cluster. The shell
script will be modified as needed. The project will also involve developing a stock 
API using flask that will provide real-time analysis of the stock market data. The performance
of the API will be compared in both Chameleon and GoogleCloud. The stock API will 
be using python, pyGal, Quandl. The benchmarking process will be registered in the project report.

\end{abstract}

\keywords{hid-sp18-417, Kubernetes, deloyment, performance}


\maketitle

\section{Introduction}

The area of research for this project is Kubernetes. Looking at the high popularity of 
the tool, it is becoming a buzz word in the world of cloud computing and containers. 
The project compares and contrasts the ease of deployment, configuration and performance 
in different systems. Before going into the detail of the project outcome, it is important
to introduce several key components of the project.

\subsection{Kubernetes}
Kubernetes is an open-source system developed and maintained by Google Inc. The system 
provides a container orchestration platform for ease of deployment, management and scaling 
of services and resources. The system places containers automatically based on it resource 
requirement not compromising availability. Similarly, it provides seamless management of 
data volumes and storage system. It allows updating configurations of the deployment without 
rebuilding the image. The system can facilitate public, private or hybrid container management 
system based on an organization’s requirement.  Kubernetes resource monitoring system oversees 
automatic rollout and rollback of the resources in case of fallbacks.


Kubernetes is also known as K8 or kube. This is originally developed based on Google’s container 
management system. A Kubernetes can be deployed in a single host, but it is designed to be
benefitted efficiently when used for a cluster of connected hosts. Use of a cluster of 
multiple hosts facilitates high availability.


The primary components of the system are: 

The master node controls the other nodes. Nodes are the hosts in the cluster. Pod is a group 
of one or more containers deployed in a node.  The Replication Controller monitors and controls
the replicas of the services. A Kubernetes service is a single unit that has a collection of 
rules/configuration for effective deployment of a container. Various metadata such as replicas,
environment, track, port, can be defined in a service configuration. Kubectl is the command line 
configuration tool provided by Kubernetes. 

The figure Figure~\ref{fig:kube-archtecture} below explains the Kubernetes architecture. The
master is responsible for managing the 
deployments, scheduling, exposing containerized applications.

\begin{figure}[htb]
	\centering\includegraphics[width=\columnwidth]
        {images/hid_417_Kubernetes-Architecture}
	\caption{Kubernetes Architecture~\cite{hid-sp18-417-kubernetes}}\label{fig:kube-archtecture}
\end{figure}

\subsection{GKE}

Google Kubernetes Engine is the tool developed by Google Inc to simplify the management and 
orchestration of Kubernetes systems, in Google’s public cloud services. This will help an
organization focus more into their ow product development than worrying about Kubernetes 
networking, upgrades and maintenance. As the main focus of the project was Kubernetes, it
was vital for the author to include GKE in the scope of the project.

Here are some benefits: With use of GKE, user need not worry about Kubernetes master. 
The system ensures that master is always up and running. User need not worry about underlying 
networking on what to user e.g. weave, flannel etc. It makes access and identity management
easier with Google’s Identity and Container Management System [IAM]. Auto scaling is easier
with just simple commands. With Kubernetes’ frequent releases its important for the system 
to stay updated always. The upgrade is easier with gcloud with just a single command rather 
than the manual overhead of porting the system step by step.

gcloud container clusters upgrade CLUSTER_NAME \  [ — cluster-version=X.Y.Z]

The project availed the free tier offer from Google to bring in the setup and configuration 
experience for GKE. As of Apr 2018, Google is offering a $300 credit to be used with in 12 months.
The free trial has certain resource usage limitations applied. Please refer to Google’s documentation
for details. But the bottom-line was that the limitation was well within the projects scope to explore
the cluster’s performance so the author decided to go ahead with the research. 

Here are the key steps involved to get you started with GKE:


\subsection{Python API}
\TODO{Python API control structure and available features are discussed}

\subsection{Flask Service}
\TODO{the details of the swagger service in use}

\subsection{setup process}
\TODO{Initial setup experience is discussed}

\subsection{Chameleon}
\TODO{JetStream setup is discussed.}

\subsection{Stocks Analysis}
\TODO{Practice use of the webservice and the current 
market trend of tools and technologies in the context is discussed.}

\subsection{Trouble shooting}
\TODO{Troubleshooting experience during the project is logged}

\subsection{BenchMarking}
\TODO{Compare and contrast the installation, setup and performance of the service in Pi and Jetstream cluster}


\section{Conclusion}

\TODO{Put here an conclusion.} Conclusion and abstracts must not have any
citations in the section.


\begin{acks}
The author would like to thank Dr. Gregor von Laszewski for his support and 
suggestions in writing this paper.
\end{acks}

\bibliographystyle{ACM-Reference-Format}
\bibliography{report}



