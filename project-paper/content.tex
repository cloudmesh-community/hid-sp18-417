% status: 2
% chapter: Security

\title{Real-time Stocks Analysis Service}

\author{Rashmi Ray}
\affiliation{%
  \institution{Indiana University}
  \streetaddress{107 S. Indiana Avenue}
  \city{Bloomington}
  \state{Indiana}
  \postcode{43017-6221}
}
\email{rashray@iu.edu}
% The default list of authors is too long for headers}

\renewcommand{\shortauthors}{Uma Kugan}
\begin{abstract}
The project involves creating a Kubernete cluster in 3 Rasberry Pis and later 
automating the installation and setup proess by including the shell script.
The same process will be excuted in a Kubernetes Jetstream process. The shell
script will be modified as needed. The project will also involve developing a stock 
API that will provide real-time analysis of the stock market data. The performance
of the API will be compared in both jetstream and pi cluster. The stock API will 
be using python, matplotlib. The benchmarking process will be registered in the project report.
\end{abstract}

\keywords{hid-sp18-417, Kubernetes, deloyment, performance}


\maketitle

\section{Introduction}
The project targets on Kubernetes container management system. As the tool is popularly used
for container management and deployment, this project looks into exploring the tool.
The primary task involves comparing and contrasting the initial installation, 
setup and rest service deployment in two separate systems.  
\TODO{Complete introduction.}



\subsection{Kubernetes}
Kubernetes is an open-source system used for automating deployment, 
scaling and management of containerized applications. It was originally 
designed by Google and now maintained by the Cloud Native Computing 
Foundation.\cite{hid-sp18-417-kubernetes}
\TODO{Kubernetes details}

\subsection{Rasberry Pi}
\TODO{Rasberry Pi setup and configuration is discussed in the context of the project}

\subsection{Python API}
\TODO{Python API control structure and available features are discussed}

\subsection{Swagger Service}
\TODO{the details of the swagger service in use}

\subsection{setup process}
\TODO{Initial setup experience is discussed}

\subsection{JetStream}
\TODO{JetStream setup is discussed.}

\subsection{Stocks Analysis}
\TODO{Practice use of the webservice and the current 
market trend of tools and technologies in the context is discussed.}

\subsection{Trouble shooting}
\TODO{Troubleshooting experience during the project is logged}

\subsection{BenchMarking}
\TODO{Compare and contrast the installation, setup and performance of the service in Pi and Jetstream cluster}


\section{Conclusion}

\TODO{Put here an conclusion.} Conclusion and abstracts must not have any
citations in the section.


\begin{acks}
The author would like to thank Dr. Gregor von Laszewski for his support and 
suggestions in writing this paper.
\end{acks}

\bibliographystyle{ACM-Reference-Format}
\bibliography{report}



