% status: 90
% chapter: IaaS

\title{Apache CloudStack}


\author{Rashmi Ray}
\affiliation{%
  \institution{Indiana University}
  \streetaddress{}
  \city{Bloomington} 
  \state{IN} 
  \postcode{47408}
}
\email{rashray@iu.edu}


% The default list of authors is too long for headers}
\renewcommand{\shortauthors}{G. v. Laszewski}


\begin{abstract}

  In this ever-evolving age of cloud computing, it is important for
  organizations to adopt the right cloud orchestration tool to promote
  the rapid growth of their infrastructure. Ease of maintenance, 
  robust research-base for future growth
  also facilitates the process. CloudStack can be the solution to 
  the problem. Apache CloudStack is a top-level project from Apache Foundation.

\end{abstract}

\keywords{hid-sp18-417, CloudStack, apache, cloud architecture, cloud
  technology, IaaS, i516}

\maketitle

\section{Introduction}

``Apache CloudStack is open source software designed to deploy and
manage large networks of virtual machines, as a highly available,
highly scalable Infrastructure as a Service (IaaS) cloud computing
platform. CloudStack is used by a number of service providers to offer
public cloud services, and by many companies to provide an on-premises
(private) cloud offering, or as part of a hybrid cloud
solution.''~\cite{hid-sp18-417-www-cloudstack-intro}. It is a Java based project.


\section{History}

CloudStack project was initiated by cloud.com, formerly known as
VMOps.  The project started in 2008 and was released under GNU general
public license in 2010.  In the same year Microsoft partnered with cloud.com to 
facilitate support of Microsoft server integration. 
In Jul 2011 Citrix purchased the software and
released it again. In 2012 Citrix renewed CloudStack license under
the Apache and later donated the project to Apache.  Since then the
tool has graduated from Apache incubator and released in Mar 2013.


\section{Key Features}

\paragraph{Hypervisor Support:} CloudStack currently supports most of 
the major hypervisors. It currently supports: XenServer with XAPI, KVM,
and VMware with vSphere.
CloudStack supports a single cloud to constitute of multiple types of
hypervisor implementations. Hosts of a single cluster must run same
software/hypervisor. This is considered a key advantage when
implementing CloudStack on an existing infrastructure to enable
maximum use of the existing resources. ~\cite{hid-sp18-417-www-cloudstack-hypervisor}.

\paragraph{Massively Scalable:}
``CloudStack can manage tens of thousands of physical servers installed
in geographically distributed datacenters. The management server
scales near-linearly eliminating the need for cluster-level management
servers. maintenance or other outages of the management server can
occur without affecting the virtual machines running in the cloud''~\cite{hid-sp18-417-www-cloudstack-scalability}.

\paragraph{High Availability:} CloudStack provides features to
increase availability. It enables the management server deployed
across multiple nodes, this increasing availability through load
balancing. MYSQL can be configured simulate a replication that can
serve as a fallback in case of data loss. The tool also provides a
{\bf robust rest API} for operation and management of the cloud. The
API can be used with a CloudStack system as the underlying server.
The software also provides Amazon Web Services compatible
API. ``CloudStack provides an EC2 API translation layer to permit the
common EC2 tools to be used in the use of a CloudStack
cloud.''~\cite{hid-sp18-417-www-cloudstack-aws} The tool also eases
configuration management, Hypervisor agnostic, user management,
snapshot management and networking resource optimization.

\paragraph{GUI:} It might not be a prominent advantage for the cloud
computing community as it mostly prefers to carryout instruction in
the console, but for occasional users it is an added benefit that
CloudStack provides an efficient customizable Graphical User
Interface. In all CloudStack provides a comprehensive package of
features that can enable an organization to deploy a full featured
cloud Infrastructure. CloudStack also offers CloudMonkey CLI 
tool to easy orchestration from console.

CloudStack offers orchestration of networking from datalink layer
to application layer through several secure channels:
firewall, Virtual Private Network, Dynamic Host Configuration 
Protocol, Network Address Translation and others.

\section{Terminology}

It is important to be aware of certain terminology to get an idea on
how resources are organized in CloudStack.
\paragraph{Management Server}:
A management server is the control system of any CloudStack deployment. 
Depending of the size or the cloud there can be multiple management 
servers deployed to ease the traffic and facilitate high availability.
A management server needs Java, Apache Tomcat and MySQL server. The server
controls the allocation of resources, manages snapshots and disk images,
provides API interface.



Grouping resources appropriately enhances availability and fault tolerance. Here is the 
list in descending order in terms of size:
\paragraph{Region:}	The largest available unit in CloudStack infrastructure is a 
region. In case of a large Cloud network it is ideal to group it into 
several geographic regions. A region may consist of multiple datacenters
located in proximity. It can be managed by one or a cluster of management
servers. Data can be spanned across multiple regions so that in case of
failure of one region the services will still be available from other regions(s).
\paragraph{Zone:}	A zone typically consists of a datacenter. A zone can have it is own LAN
network and supply line. 
\paragraph{Pod:}	A pod is typically a rack or cluster(s) of hosts. Hosts in a Pod are 
expected to be in same subnet. 
\paragraph{Cluster:} 	A cluster is a group of hosts that can be a pool of VMwares or KVM 
servers. ``The hosts in a cluster all have identical hardware, run the same 
hypervisor, are on the same subnet, and access the same shared primary storage.
Virtual machine instances (VMs) can be live-migrated from one host to another
within the same cluster, without
interrupting service to the user'' ~/cite{hid-sp18-417-www-cloudstack-management-cluster}
\paragraph{Host:}	A host is a single node/machine that has hypervisor installed.
\paragraph{Storage:}	At least one Primary Storage is associated with a cluster or zone. The 
storage is available to each cluster in the zone and store a virtual copy of 
each VMs to be used in case of failure. CloudStack also offers a secondary storage
option for ISO images and snapshots.
Figure~\ref{F:cloudstack-resource-group} depicts how resource can 
be grouped in a CloudStack deployment.

\begin{figure}[htb]
\includegraphics[width=\columnwidth]{images/hid-sp18-417-cloudstack.png}
\caption{CloudStack Resource Organization~\cite{hid-sp18-417-cloudstack-resource-grouping}}
\label{F:cloudstack-resource-group}
\end{figure}

\section{Deployment Architecture}

To use CloudStack, several deployment architectures can be adopted
depending on the cloud requirements of an organization. The strength
of the deployment can range from a single machine to thousands of
nodes spread across the globe in several data centers using various supported
networking technologies.

\paragraph {Networking:} Though CloudStack support many types of networking 
technologies; they typically fall into two categories: Basic networking uses 
Flat layer 2 network design. This is similar to that AWS-Classic uses. The
guest users are isolated at layer 3 using a hypervisor bridge device. Here 
every machine can communicate to others. The devices are hosted typically in 
the same subnet.  This design is suitable for simpler subnet that ensures 
reliable firewall. For sophisticated deployments, advanced networking is 
adopted. This implementation uses layer 2 isolation design such as SDN technologies.
This enhances security and availability.~\cite{hid-sp18-417-www-cloudstack-networking} 

\paragraph {A CloudStack deployment} primarily consists of a management server and
set of resources [IP addresses, VLANs, storage systems] to constitute
the cloud.  For simplest form of deployment, only one single machine
can be used to serve both as the management server and hypervisor
host. The same system can be scaled to feed increased demand to spread
across multiple management servers to support larger userbase and
facilitate better load balancing. ``The management server typically runs
on a dedicated machine or as a virtual machine. It controls allocation
of virtual machines to hosts and assigns storage and IP addresses to
the virtual machine instances. The Management Server runs in an Apache
Tomcat container and requires a MySQL database for
persistence.''~\cite{hid-sp18-417-www-cloudstack-management-server} 
The management server as the name suggests provides the single point 
Management controlling the configurations, snapshots, resources monitoring, 
API, Web Interface etc. 

\paragraph {Pluggable Model:} CloudStack supports a pluggable model. It 
allows the organization to cherry-pick the desired features. ``Customers may 
choose SDN support from Nicira, Midokura or Big Switch Networks and choose load
balancing using NetScaler or F5. For object storage, CloudStack offers a choice of
products from Basho, Caringo, Cloudian or EMC's Atmos. Users also may choose
OpenStack Swift component.'' ~\cite{hid-sp18-417-www-cloudstack-model}
This model provides flexibility for the organizations to scale their infrastructure 
using latest technology as well as reusing the existing resources.

\section{Installation}

Before the CloudStack installation it is important to decide on the resources,
deployment architecture-based on resources, hypervisors, network and storage 
configurations. The management server needs Cent OS or ubuntu operating system
with 64bit x86 processor. The server needs a minimum of 4 GB RAM and 250 GB
hard disk space. 

\paragraph {Hypervisor Setup:} Each of the host must have hypervisor installed that is Hardware
Virtual Machine[HVM] supported. A host optimally needs 64bit x86 processor, 4 GB
RAM, 36 GB hard disk. It is worth noting that while deploying CloudStack the Virtual 
Machine shouid not be running. Each of the machines needs at least 1 Network Interface Card 
to connect to the cloud network.  After the OS installation vh-util need to be installed 
in case Xen server. Now is time to install and configure the very first management 
server following by setting up the mysql DB. Once the DB is setup it is important to 
setup Primary and Secondary storage for high availability. Finally, the system virtual 
machine template needs to be prepared for the secondary storage. Please note
 that additional management servers can be added if required. It is an optional configuration.

\paragraph {Package Repository}: Officially, CloudStack offers distribution source from the 
official mirror. Deployment from the source is an elaborate process. To ease the installation 
process for simpler systems, many opensource community members provide convenient binaries
so that the user doesn’t have to install from the source going through a detailed setup 
process. The packaged repositories are typically helpful for trial user. ~\cite{hid-sp18-417-www-cloudstack-installation}


\section{Best Practices}

The official website of CloudStack recommends a set of best practices for ease of 
maintenance and installation. It is advised that a full-scale production deployment 
can take 4-8 weeks, so the deployment to be planned accordingly for possible 
fallbacks and troubleshooting. It always eases the deployments issues if a sophisticated 
advanced system deployment is done after a staging. A smaller system that emulates the production 
 will help the deployment team foresee the possible hurdles well in advance. 
For security, each machine should be allowed contact 
with limited resources and over a secure SSL channel. RAM and disk storage usage monitoring 
is the key to promote high availability. It is ideal to disable allocation of a hypervisor
to a cluster when the max limit is approaching. Regular application of hotfixes is also necessary for 
seamless operation. It is better to use smaller multiple storages per cluster than a big single one.
This promotes high availability. Number of virtual machines to be allocated per hypervisor to be limited per the
specifications of the type of the hypervisor used. It is better to allow a safety margin to handle the unforeseen.
 ~\cite{hid-sp18-417-www-cloudstack-best-practices}


\section{Conclusion}

\paragraph {Community:} 
Regular CloudStack events are being conducted all over the world to promote team c
ollaborations, meetups and contributions. CloudStack Collaboration Conference is 
one of the major events across major cities that the developers look forward to. 
Datapipe is one of the major user of CloudStack. The major users of CloudStack
includes: Apple , Citrix Systems, Dell, Disney, Hitachi , Juniper Networks, PPTV, SAP, 
WebMD etc. 

CloudStack provides a comprehensive package of unique features for an 
organization to develop a robust cloud architecture. The platform being an open source project
from Apache foundation, it provides a regular stable updates and releases that enables it to 
stay in put with the latest in the industry. Even though it provides a comparatively smoother deployment,
CloudStack installation still involves a time-intensive process with several steps. Though 
availability of package repository eases the process, the users are still 
hoping the process to get better with newer versions of the tool. 
There is a good developer community support
behind CloudStack. Apache normally releases a long term release every two years.
 There can be short term releases for bug fixes or security patches.
The most recent source release of CloudStack is 4.11.0.0.

\paragraph {Security:} CloudStack has a dedicated security team to identify possible vulnerabilities 
and address to breach reports. A possible exposure can be reported at security@cloudstack.apache.org
with the details.  


\begin{acks}

  The author would like to thank Dr.~Gregor~von~Laszewski for his
  support and suggestions to write this paper.

\end{acks}

\bibliographystyle{ACM-Reference-Format}
\bibliography{report} 
