% status: 0
% chapter: TBD

\title{Apache CloudStack an Overview}


\author{Rashmi Ray}
\affiliation{%
  \institution{Indiana University}
  \streetaddress{}
  \city{Bloomington} 
  \state{IN} 
  \postcode{47408}
}
\email{rashray@iu.edu}

\author{Gregor von Laszewski}
\affiliation{%
  \institution{Indiana University}
  \streetaddress{Smith Research Center}
  \city{Bloomington} 
  \state{IN} 
  \postcode{47408}
  \country{USA}}
\email{laszewski@gmail.com}


% The default list of authors is too long for headers}
\renewcommand{\shortauthors}{G. v. Laszewski}


\begin{abstract}
In this ever-evolving age of cloud computing, it’s important for organizations to adopt the right
cloud orchestration tool to support the rapid growth of their infrastructure. Ease of scalability, robust research-base
for future growth and current large userbase also facilitates the process. CloudStack can be the 
answer here. Apache CloudStack is a top level project from Apache, so it indeed has a support of a solid team of researchers.

\end{abstract}

\keywords{hid-sp18-417, CloudStack, apache, cloud architecture, cloud technology, i516}

\maketitle

\section{Introduction}
‘‘Apache CloudStack is open source software designed to deploy and manage large networks of 
virtual machines, as a highly available, highly scalable Infrastructure as a Service (IaaS)
cloud computing platform. CloudStack is used by a number of service providers to offer public 
cloud services, and by many companies to provide an on-premises (private) cloud offering, 
or as part of a hybrid cloud solution.’’~\cite{hid-sp18-417-www-cloudstack-intro}. 
Infrastructure as a Service[IaaS] is kind of cloud computing service that enables cloud 
configuration over the internet. 


\section{History}
The CloudStack was originally developed by cloud.com formerly know as VMOs. 
The project started in 2008 and was released under GNU general public license. 
In Jul 2011 Citrix purchased the software and released it again. In 2012 Citrix 
renewed CloudStack licensed under the Apache and later donated the project to Apache.
Since then The tool has graduated from Apache incubator and released in Mar 2013. 

\section{Organization}

\section{Deployment Architecture}

\section{Terminology}

\section{Key Features}

\section{Growth}

\section{Conclusion}

\bibliographystyle{ACM-Reference-Format}
\bibliography{report} 
