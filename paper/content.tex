% status: 0
% chapter: TBD

\title{Apache CloudStack an Overview}


\author{Rashmi Ray}
\affiliation{%
  \institution{Indiana University}
  \streetaddress{}
  \city{Bloomington} 
  \state{IN} 
  \postcode{47408}
}
\email{rashray@iu.edu}

\author{Gregor von Laszewski}
\affiliation{%
  \institution{Indiana University}
  \streetaddress{Smith Research Center}
  \city{Bloomington} 
  \state{IN} 
  \postcode{47408}
  \country{USA}}
\email{laszewski@gmail.com}


% The default list of authors is too long for headers}
\renewcommand{\shortauthors}{G. v. Laszewski}


\begin{abstract}

  In this ever-evolving age of cloud computing, it is important for
  organizations to adopt the right cloud orchestration tool to support
  the rapid growth of their infrastructure. Ease of scalability,
  robust research-base for future growth and current large userbase
  also facilitates the process. CloudStack can be the answer
  here. Apache CloudStack is a top level project from Apache, so it
  indeed has a support of a solid team of researchers.

\end{abstract}

\keywords{hid-sp18-417, CloudStack, apache, cloud architecture, cloud
  technology, i516}

\maketitle

\section{Introduction}

``Apache CloudStack is open source software designed to deploy and
manage large networks of virtual machines, as a highly available,
highly scalable Infrastructure as a Service (IaaS) cloud computing
platform. CloudStack is used by a number of service providers to offer
public cloud services, and by many companies to provide an on-premises
(private) cloud offering, or as part of a hybrid cloud
solution.''~\cite{hid-sp18-417-www-cloudstack-intro}.  Infrastructure
as a Service[IaaS] is kind of cloud computing service that enables
cloud configuration over the internet.


\section{History}

The CloudStack was originally developed by cloud.com formerly know as
VMOs.  The project started in 2008 and was released under GNU general
public license.  In Jul 2011 Citrix purchased the software and
released it again. In 2012 Citrix renewed CloudStack licensed under
the Apache and later donated the project to Apache.  Since then The
tool has graduated from Apache incubator and released in Mar 2013.


\section{Key Features}

\paragraph{Hypervisor Support:} A hypervisor or Virtual Machine
Monitor is a software or firmware that creates and runs virtual
machines. CloudStack currently supports most of the major hypervisors.
CloudStack enables a single cloud to constitute of multiple types of
hypervisor implementations. This is considered a key advantage when
implementing CloudStack on an existing infrastructure to enable
maximum use of the existing resources. {\bf Massively Scalable:}
CloudStack can manage tens of thousands of physical servers installed
in geographically distributed datacenters. The management server
scales near-linearly eliminating the need for cluster-level management
servers. Maintenance or other outages of the management server can
occur without affecting the virtual machines running in the cloud
~\cite{hid-sp18-417-www-cloudstack-scalability}.

\paragraph{High Availability:} CloudStack provides features to
increase availability. It enables the management server deployed
across multiple nodes, this increasing availability through load
balancing. MYSQL can be configured simulate a replication that can
serve as a fallback in case of data loss. The tool also provides a
{\bf robust rest API} for operation and management of the cloud. The
API can be used with a CloudStack system as the underlying server.
The software also provides Amazon Web Services compatible
API. CloudStack provides an EC2 API translation layer to permit the
common EC2 tools to be used in the use of a CloudStack
cloud. ~\cite{hid-sp18-417-www-cloudstack-aws} The tool also eases
configuration management, Hypervisor agnostic, user management,
snapshot management and networking resource optimization.

\paragraph{GUI:} It might not be a prominent advantage for the cloud
computing community as it mostly prefers to carryout instruction in
the console, but for occasional users it is an added benefit that
CloudStack provides an efficient customizable Graphical User
Interface. In all CloudStack provides a comprehensive package of
features that can enable an organization to deploy a full featured
cloud Infrastructure.

\section{Terminology}

\section{Deployment Architecture}

To use CloudStack, several deployment architectures can be adopted
depending on the cloud requirements of an organization. The strength
of the deployment can range from a single machine to thousands of
nodes spread across the globe in several data centers several
networking technologies.

A CloudStack deployment primarily consists of a management server and
set of resources [IP addresses, VLANs, storage systems] to constitute
the cloud.  For simplest form of deployment, only one single machine
can be used to serve both as the management server and hypervisor
host. The same system can be scaled to feed increased demand to spread
across multiple management servers to support larger userbase and
facilitate better load balancing. The management server typically runs
on a dedicated machine or as a virtual machine. It controls allocation
of virtual machines to hosts a nd assigns storage and IP addresses to
the virtual machine instances. The Management Server runs in an Apache
Tomcat container and requires a MySQL database for
persistence. ~\cite{hid-sp18-417-www-cloudstack-management-server}



\section{Growth}

\section{Conclusion}

\begin{acks}

  The author would like to thank Dr.~Gregor~von~Laszewski for his
  support and suggestions to write this paper.

\end{acks}

\bibliographystyle{ACM-Reference-Format}
\bibliography{report} 
