% status: 0
% chapter: TBD

\title{Apache CloudStack an Overview}


\author{Rashmi Ray}
\affiliation{%
  \institution{Indiana University}
  \streetaddress{}
  \city{Bloomington} 
  \state{IN} 
  \postcode{47408}
}
\email{rashray@iu.edu}

\author{Gregor von Laszewski}
\affiliation{%
  \institution{Indiana University}
  \streetaddress{Smith Research Center}
  \city{Bloomington} 
  \state{IN} 
  \postcode{47408}
  \country{USA}}
\email{laszewski@gmail.com}


% The default list of authors is too long for headers}
\renewcommand{\shortauthors}{G. v. Laszewski}


\begin{abstract}

\end{abstract}

\keywords{hid-sp18-417, CloudStack, apache, cloud architecture, cloud technology, i516}

\maketitle

In this ever-evolving age of cloud networks, it’s important for organizations to adopt the right tools to support the rapid growth of their infrastructure. Ease of scalability, robust research-base for future growth and current large userbase also facilitates the process. CloudStack can be the answer here. ‘‘Apache CloudStack is a top-level project of the Apache Software Foundation (ASF). The project develops open source software for deploying public and private Infrastructure-as-a-Service (IaaS) clouds.’’ ~\cite{hid-sp18-417-www-cloudstack-apache} 

\section{Introduction}
‘‘Apache CloudStack is open source software designed to deploy and manage large networks of 
virtual machines, as a highly available, highly scalable Infrastructure as a Service (IaaS)
cloud computing platform. CloudStack is used by a number of service providers to offer public 
cloud services, and by many companies to provide an on-premises (private) cloud offering, 
or as part of a hybrid cloud solution.’’~\cite{hid-sp18-417-www-cloudstack-intro}. 

\section{History}

\section{Organization}

\section{Deployment Architecture}

\section{Terminology}

\section{Key Features}

\section{Growth}

\section{Conclusion}

\bibliographystyle{ACM-Reference-Format}
\bibliography{report} 
