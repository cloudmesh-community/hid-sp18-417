\section{Apache Delta Cloud}

Apache DeltaCloud was developed in collaboration between Apache
Foundation and Redhat to provide a programming application that will
facilitate management of different cloud interfaces and It was
supporting all the major cloud interfaces.  ‘‘Each
Infrastructure-as-a-Service cloud existing today[when?] provides its
own API. The purpose of Deltacloud is to provide one unified
REST-based API that can be used to manage services on any cloud. Each
particular cloud is controlled through an adapter called a
"driver". As of June 2012, drivers exist for the following cloud
platforms: Amazon EC2, Fujitsu Global Cloud Platform, GoGrid,
OpenNebula, Rackspace, RHEV-M, RimuHosting, Terremark and VMware
vCloud’’ ~\cite{hid-sp18-417-wiki-deltacloud}.

In 2009, DeltaCloud was developed for the purpose of providing one
unified API for the major cloud service.

In 2011, it became a part of the Apache’s top level project. 

Unfortunately, in 2015 the project was closed due to inactivity.  The
application though inactive is chosen for the study to understand the
case behind the termination of the project.  It is primarily because
of lack of popularity RedHat withdrew the sponsorship ultimately
resulting in the termination of the project .
